\documentclass[aspectratio=169]{beamer}

\usetheme{default}
\setbeamertemplate{navigation symbols}{}
\setbeamertemplate{itemize item}{\color{black}\textbullet}
\setbeamertemplate{itemize subitem}{\color{black}\textbullet}
\usepackage{xcolor}
\definecolor{navy}{RGB}{0, 0, 128}

\begin{document}

\begin{frame}
Recall the inclusive value term from the bus choice model:

\begin{align*}
I_{iB}&=\log\left(1+\exp\left(\frac{u_{iRB|B}}{\lambda}\right)\right)
\end{align*}

\bigskip{}

$\lambda I_{iB} + \gamma$  is the expected utility of riding a bus, from the standpoint of the top-level nest

\bigskip{}

Can think of it as the expected value over the $\lambda \epsilon_{ij}$'s within the nest

\end{frame}

\begin{frame}

\bigskip{}

Now we can make an analogy to a dynamic model (with no discounting):

\bigskip{}

\onslide<2->{
First period: 
\begin{itemize}
    \item[]
    \item Choose bus vs car with extreme value errors for both options
    \item[]
    \item Individuals account for future choice $\epsilon$'s if they choose bus (option value)
\end{itemize}
}
\bigskip{}

\onslide<3->{
Second period: 
\begin{itemize}
    \item[]
    \item Errors distributed TIEV, independent from each other and first period errors
    \item[]
    \item Expected value of second period decision is $\lambda I_{iB}$ plus Euler's constant
\end{itemize}
}


\end{frame}

\end{document}