\documentclass[aspectratio=169]{beamer}

\usetheme{default}
\setbeamertemplate{navigation symbols}{}
\setbeamertemplate{itemize item}{\color{black}\textbullet}
\setbeamertemplate{itemize subitem}{\color{black}\textbullet}

\begin{document}



\begin{frame}

Policy experiments are where our model meets the real world

\bigskip{}
\bigskip{}


These experiments are the whole point of estimating the structural model
\begin{itemize}
\item[]
\item Structural estimation $\implies$ recovering the DGP of the model
\item[]
\item Once we know the DGP, we can simulate data from it and do policy experiments
\item[]
\item We can predict the effects of proposed or hypothetical policies
\item[]
\item Quantify the exact mechanisms through which policies affect outcomes of interest
\end{itemize}


\end{frame}



\begin{frame}
Why we should do structural estimation, according to Michael Keane:

\begin{itemize}
\item[]
\item Examine effects of policies not yet implemented
\item[]
\item Learn more about economics by looking through the lens of a model
\item[]
\item Assess performance of theoretical models in explaining real-world data
\item[]
\item Gradually build up long-run ``canonical'' models of behavior in many areas
\item[]
\item Enjoy the rigor of more complicated econometrics
\item[]
\item Observational data is much cheaper to collect than experimental data
\end{itemize}
\end{frame}



\begin{frame}
Why \textit{not} do structural estimation, according to me:

\begin{itemize}
\item[]
\item Difficult to develop and estimate a tractable yet realistic model
\item[]
\item Understanding identification takes more effort
\item[]
\item The complicated econometrics can be so challenging as to not be any fun at all
\item[]
\item Much longer time to program and obtain estimates
\item[]
\item Things are already pretty complicated even in this overly stylized model of schooling choice
\end{itemize}
\end{frame}


\end{document}