\documentclass[aspectratio=169]{beamer}

\usetheme{default}
\setbeamertemplate{navigation symbols}{}
\setbeamertemplate{itemize item}{\color{black}\textbullet}
\setbeamertemplate{itemize subitem}{\color{black}\textbullet}

\begin{document}



\begin{frame}

Most structural models require \textit{nonlinear estimation}

\begin{itemize}
\item[]
\item e.g. MLE, GMM or their simulated counterparts
\item[]
\item In nonlinear optimization, starting values are crucial
\item[]
\item Initializing at random starting values is likely to give poor results
\end{itemize}

    
\end{frame}




\begin{frame}
It's important to start simple

\begin{itemize}
\item[]
\item Try to calibrate values of all intercept parameters $(\beta_0,\alpha_0,\gamma_0)$
\item[]
\item See if you can get them to match avg. log wages, avg. schooling rate
\item[]
\item Then add more parameters and compare with more moments of the data
\item[]
\item This helps with starting values as well as verifying estimation performance
\end{itemize}


\end{frame}

\begin{frame}
Other tips:

\begin{itemize}
\item[]
\item Estimate a model that only has intercepts in each equation
\item[]
\item Verify estimates are identical across computing software products (e.g. Julia, R)
\item[]
\item Look at code from similar papers that have been published
\end{itemize}


\end{frame}

\end{document}