\documentclass[aspectratio=169]{beamer}

\usetheme{default}
\setbeamertemplate{navigation symbols}{}
\setbeamertemplate{enumerate item}{\color{navy}\arabic{enumi}.}
\setbeamertemplate{itemize item}{\color{black}\textbullet}
\setbeamertemplate{itemize subitem}{\color{black}\textbullet}
\usepackage{booktabs}
\usepackage{xcolor}
\usepackage{tikz}
\usetikzlibrary{shapes,arrows,positioning}
\definecolor{navy}{RGB}{0, 0, 128}
\definecolor{lightblue}{RGB}{230,240,250}
\definecolor{darkgreen}{RGB}{0,100,0}
\definecolor{lightgreen}{RGB}{230,250,230}
\newcommand{\highlight}[1]{\colorbox{lightblue}{$\displaystyle\textcolor{navy}{#1}$}}
\newcommand{\highlighttext}[1]{\colorbox{lightblue}{\textcolor{navy}{#1}}}
\newcommand{\highlightgreen}[1]{\colorbox{lightgreen}{$\displaystyle\textcolor{darkgreen}{#1}$}}

\begin{document}

\begin{frame}

So far: matching model moments to data

\bigskip

Logic: if the model matches the data, then it is a reasonable model

\bigskip

\onslide<2->{
Another alternative: \textcolor{navy}{Indirect Inference}

\bigskip

Uses an \textcolor{navy}{auxiliary model} as a lens through which to view the world
}

\end{frame}

\begin{frame}

\onslide<1->{
The auxiliary model doesn't need to accurately describe the DGP
}

\bigskip

\onslide<2->{
It simply acts as a lens through which to view the world
}

\bigskip

\onslide<3->{
\textcolor{navy}{Objective:} choose parameters of structural model such that:

\bigskip

simulated data = real data \textit{through the lens of the auxiliary model}
}

\end{frame}



\begin{frame}

Revisit Rust (1987) bus engine model:

\bigskip

State: mileage $X_t$

\bigskip

Flow payoffs:
\begin{align*}
u(X_t,d_t,\theta)=\left\{\begin{array}{ll}
-c(X_t,\theta) & \text{if } d_t=0 \text{ (keep)}\\
-RC & \text{if } d_t=1 \text{ (replace)}
\end{array}\right.
\end{align*}

\bigskip

\onslide<2->{
where $RC = \overline{P}-\underline{P}+c(0,\theta)$ is replacement cost
}

\bigskip

\onslide<3->{
Bellman equation:
\begin{align*}
V(X_t;\theta) &= \max_{d_t} \Big\{ u(X_t,d_t;\theta) + \beta \mathbb{E}[V(X_{t+1};\theta)|X_t,d_t] \Big\}
\end{align*}

}

\end{frame}






\begin{frame}

\onslide<1->{
\textcolor{navy}{Challenge:} Can't easily invert auxiliary statistics to get $\theta$

\bigskip
}

\onslide<2->{
Even with nested fixed point algorithm, alternative estimation approaches useful:
\bigskip\par
}
\begin{itemize}
\itemsep1.5em
\item<3-> Auxiliary model may be simpler to estimate
\item<4-> Can use multiple auxiliary statistics simultaneously
\item<5-> Robust to auxiliary model misspecification
\end{itemize}

\end{frame}




\begin{frame}

\textcolor{navy}{Step 1:} Choose auxiliary model

\bigskip

\onslide<2->{
Example: Logit for replacement probability
$$P(d_t=1|X_t) = \frac{\exp(\alpha_0 + \alpha_1 X_t)}{1+\exp(\alpha_0 + \alpha_1 X_t)}$$
}

\bigskip

\onslide<3->{
Estimate on real data: $\hat{\alpha} = (\hat{\alpha}_0, \hat{\alpha}_1)$
}

\end{frame}

\begin{frame}

\textcolor{navy}{Step 2:} For candidate $\theta$, simulate the structural model

\bigskip

\onslide<2->{
\begin{itemize}
\itemsep1.5em
\item<2-> Solve dynamic program to get $V(x;\theta)$ and policy $\Pr(d^*=1\vert x;\theta)$
\item<3-> Simulate $S$ bus histories using policy and transitions
\item<4-> Each simulation: $\{X_t^s, d_t^s\}_{t=1}^T$ for $s=1,\ldots,S$
\end{itemize}
}

\end{frame}

\begin{frame}

\textcolor{navy}{Step 3:} Estimate auxiliary model on simulated data

\bigskip

\onslide<2->{
Run same logit on simulated data: get $\tilde{\alpha}(\theta)$

\bigskip
}

\onslide<3->{
This is the \textcolor{navy}{binding function}: $\theta \mapsto \tilde{\alpha}(\theta)$

\bigskip
}

\onslide<4->{
No closed form! Must simulate for each candidate $\theta$
}

\end{frame}

\begin{frame}

\textcolor{navy}{Step 4:} Minimize distance between auxiliary parameters in real and simulated data

\bigskip

\onslide<2->{
$$\hat{\theta} = \arg\min_\theta \left[\hat{\alpha} - \tilde{\alpha}(\theta)\right]' W \left[\hat{\alpha} - \tilde{\alpha}(\theta)\right]$$
}

\bigskip

\onslide<3->{
where $W$ is a weighting matrix (often $\widehat{Var}(\hat{\alpha})^{-1}$)
}

\end{frame}

\begin{frame}

\onslide<1->{
Key insight: auxiliary model captures reduced-form patterns

\bigskip
}

\onslide<2->{
If structural model is correct:
\bigskip\par

\begin{itemize}
\itemsep1.5em
\item<3-> Real data generated by true $\theta_0$
\item<4-> Simulated data from $\theta_0$ should have same auxiliary statistics
\item<5-> $\hat{\alpha} \approx \tilde{\alpha}(\theta_0)$ when we find the right $\theta_0$
\end{itemize}
}

\end{frame}

\begin{frame}

Could use multiple auxiliary statistics:

\bigskip

\onslide<2->{
\begin{enumerate}
\itemsep1.5em
\item<2-> Logit coefficients $(\alpha_0, \alpha_1)$
\item<3-> Mean replacement mileage $\mathbb{E}[X_t | d_t=1]$
\item<4-> Replacement hazard rate at various mileages
\item<5-> Mean time between replacements
\end{enumerate}
}

\bigskip

\onslide<6->{
Stack all into vector $\alpha$, estimate $\hat{\alpha}$ and $\tilde{\alpha}(\theta)$
}

\end{frame}

\begin{frame}

\begin{itemize}
\itemsep1.5em
\item<1-> Simulation-based: handles complicated dynamics easily
\item<2-> Doesn't require specifying likelihood (contrast with MLE)
\item<3-> Robust: auxiliary model need not be correctly specified
\item<4-> Flexible: can use multiple auxiliary models
\item<5-> Intuitive: match reduced-form patterns (2SLS, DiD, ...)
\end{itemize}

\end{frame}

\begin{frame}

\onslide<1->{
For each $\theta$ in optimization:
\bigskip\par

\begin{enumerate}
\itemsep1.5em
\item<2-> Solve dynamic program (NFXP or other method)
\item<3-> Simulate $N$ buses for $T$ periods, repeat $S$ times
\item<4-> Estimate auxiliary model on each simulation, average: $\bar{\tilde{\alpha}}(\theta) = \frac{1}{S}\sum_{s=1}^S \tilde{\alpha}_s(\theta)$
\item<5-> Compute distance $||\hat{\alpha} - \bar{\tilde{\alpha}}(\theta)||$
\end{enumerate}
}

\bigskip

\onslide<6->{
Large $S$ reduces simulation noise in $\bar{\tilde{\alpha}}(\theta)$; large $N$ improves precision of each $\tilde{\alpha}_s(\theta)$
}

\end{frame}



\end{document}