\documentclass[aspectratio=169]{beamer}

\usetheme{default}
\setbeamertemplate{navigation symbols}{}
\setbeamertemplate{itemize item}{\color{black}\textbullet}
\setbeamertemplate{itemize subitem}{\color{black}\textbullet}
\setbeamertemplate{enumerate item}{\color{navy}\arabic{enumi}.}
\usepackage{xcolor}
\definecolor{navy}{RGB}{0, 0, 128}
\definecolor{lightblue}{RGB}{230,240,250}
\newcommand{\highlight}[1]{\colorbox{lightblue}{$\displaystyle\textcolor{navy}{#1}$}}
\newcommand{\highlighttext}[1]{\colorbox{lightblue}{\textcolor{navy}{#1}}}

\begin{document}


\begin{frame}

Finite horizon backward recursion

\bigskip{}

Period $T$:
\begin{align*}
v_{ijT\phantom{-1}} = u_{ijT\phantom{-1}} \phantom{ + \beta \mathbb{E}\max_{k}\left\{v_{ikT-1}+\epsilon_{ikT-1}|d_{iT-2}=j\right\}}
\end{align*}


\onslide<2->{
\bigskip{}

Period $T-1$:
\begin{align*}
v_{ijT-1} = u_{ijT-1} + \beta \mathbb{E}\max_{k}\left\{u_{ikT\phantom{-1}}+\epsilon_{ikT\phantom{-1}}|d_{iT-1}=j\right\}
\end{align*}
}

\onslide<3->{
\bigskip{}

Period $T-2$:
\begin{align*}
v_{ijT-2} = u_{ijT-2} + \beta \mathbb{E}\max_{k}\left\{v_{ikT-1}+\epsilon_{ikT-1}|d_{iT-2}=j\right\}
\end{align*}
}

\onslide<4->{
\bigskip{}

General form:
\begin{align*}
v_{ijt} = u_{ijt} + \beta \mathbb{E}\max_{k}\left\{v_{ikt+1}+\epsilon_{ikt+1}|d_{it}=j\right\}
\end{align*}
}

\end{frame}




\begin{frame}

Let $f_{jt}(X_{it+1}|X_{it})$ be transition density for choice $j$

\onslide<2->{
\bigskip{}

\textcolor{navy}{Example}: College major choice
\bigskip{}

\begin{itemize}
\setlength\itemsep{1.25em}
\item Choice set: \{Physics, Literature, Other\}
\item $X_{it}$: GPA, completed courses, job market conditions
\item If you choose physics $\to$ $GPA_{i,t+1}$ may go down (due to harsh grading)
\item If you choose literature $\to$ $GPA_{i,t+1}$ may go up (due to lenient grading)
\end{itemize}
}

\onslide<3->{
\bigskip{}

Key insight: Today's major choice changes tomorrow's opportunities
\bigskip{}

Past choices create path dependence through skill accumulation
}
\end{frame}





\begin{frame}
\bigskip{}

Conditional value function:

\bigskip{}

\only<1>{
\begin{align*}
v_{jt}(X_{it}) &= u_{jt}(X_{it}) + \beta \int \mathbb{E}_{\epsilon}\left\{\max_{k} v_{kt+1}(X_{it+1})+\epsilon_{ikt+1}\right\}dF_{jt}(X_{it+1}|X_{it})
\end{align*}
\bigskip{}

Let's break down each component:

\bigskip{}
}

\only<2>{
\begin{align*}
\highlight{v_{jt}(X_{it})} &= u_{jt}(X_{it}) + \beta \int \mathbb{E}_{\epsilon}\left\{\max_{k} v_{kt+1}(X_{it+1})+\epsilon_{ikt+1}\right\}dF_{jt}(X_{it+1}|X_{it})
\end{align*}
\bigskip{}

\phantom{Let's break down each component:}

\bigskip{}

\textcolor{navy}{$v_{jt}(X_{it})$}: Conditional value of choosing $j$ today given state $X_{it}$
}

\only<3>{
\begin{align*}
v_{jt}(X_{it}) &= \highlight{u_{jt}(X_{it})} + \beta \int \mathbb{E}_{\epsilon}\left\{\max_{k} v_{kt+1}(X_{it+1})+\epsilon_{ikt+1}\right\}dF_{jt}(X_{it+1}|X_{it})
\end{align*}
\bigskip{}

\phantom{Let's break down each component:}

\bigskip{}

\textcolor{navy}{$u_{jt}(X_{it})$}: Flow utility today from choosing $j$
}

\only<4>{
\begin{align*}
v_{jt}(X_{it}) &= u_{jt}(X_{it}) + \highlight{\beta} \int \mathbb{E}_{\epsilon}\left\{\max_{k} v_{kt+1}(X_{it+1})+\epsilon_{ikt+1}\right\}dF_{jt}(X_{it+1}|X_{it})
\end{align*}
\bigskip{}

\phantom{Let's break down each component:}

\bigskip{}

\textcolor{navy}{$\beta$}: Discount factor
}

\only<5>{
\begin{align*}
v_{jt}(X_{it}) &= u_{jt}(X_{it}) + \beta \highlight{\int} \mathbb{E}_{\epsilon}\left\{\max_{k} v_{kt+1}(X_{it+1})+\epsilon_{ikt+1}\right\}dF_{jt}(X_{it+1}|X_{it})
\end{align*}
\bigskip{}

\phantom{Let's break down each component:}

\bigskip{}

\textcolor{navy}{$\displaystyle\int$}: Integral over all possible future states $X_{it+1}$
}

\only<6>{
\begin{align*}
v_{jt}(X_{it}) &= u_{jt}(X_{it}) + \beta \int \highlight{\mathbb{E}_{\epsilon}\left\{\max_{k} v_{kt+1}(X_{it+1})+\epsilon_{ikt+1}\right\}} dF_{jt}(X_{it+1}|X_{it})
\end{align*}
\bigskip{}

\phantom{Let's break down each component:}

\bigskip{}

\textcolor{navy}{$\displaystyle\mathbb{E}_{\epsilon}\left\{\max_{k} v_{kt+1}(X_{it+1})+\epsilon_{ikt+1}\right\}$}: Expected future value (integrated over pref. shocks)
}

\only<7>{
\begin{align*}
v_{jt}(X_{it}) &= u_{jt}(X_{it}) + \beta \int \mathbb{E}_{\epsilon}\left\{\max_{k} v_{kt+1}(X_{it+1})+\epsilon_{ikt+1}\right\}\highlight{dF_{jt}(X_{it+1}|X_{it})}
\end{align*}
\bigskip{}

\phantom{Let's break down each component:}

\bigskip{}

\textcolor{navy}{$dF_{jt}(X_{it+1}|X_{it})$}: State transition probability (depends on today's choice $j$)
\bigskip{}

Also assumes \textcolor{navy}{Markov property}: future states depend only on current state and choice, not entire past history
\bigskip{}

Notation: $\displaystyle \textcolor{navy}{dF_{jt}(X_{it+1}|X_{it})} = f_{jt}(X_{it+1}|X_{it})dX_{it+1}$ where $f(\cdot)$ is the PDF
}

\only<8>{
\begin{align*}
v_{jt}(X_{it}) &= u_{jt}(X_{it}) + \beta \int \mathbb{E}_{\epsilon}\left\{\max_{k} v_{kt+1}(X_{it+1})+\epsilon_{ikt+1}\right\}dF_{jt}(X_{it+1}|X_{it})
\end{align*}
\bigskip{}


Note that the integral could turn into a summation if we discretize $F_{jt}\left(\cdot\right)$
\bigskip{}

}

\end{frame}






\begin{frame}
What is the formula for $\displaystyle\mathbb{E}_{\epsilon}\left\{\max_{k} v_{kt+1}(X_{it+1})+\epsilon_{ikt+1}\right\}$? Depends on distr. of $\epsilon$'s

\bigskip{}

\onslide<2->{
T1EV case:
}
\onslide<3->{
\begin{align*}
    \mathbb{E}_{\epsilon}\left\{\max_{k} v_{kt+1}(X_{it+1})+\epsilon_{ikt+1}\right\} &= \log\left(\sum_k \exp\left(v_{kt+1}(X_{it+1})\right)\right)+\underbrace{c}_{\text{Euler's\,\,constant}}
\end{align*}
}

\onslide<4->{
General GEV case:
}
\onslide<5->{
\begin{align*}
    \mathbb{E}_{\epsilon}\left\{\max_{k} v_{kt+1}(X_{it+1})+\epsilon_{ikt+1}\right\} &= \log\left(G\left(\exp\left(v_{kt+1}\left(X_{it+1}\right)\right)\right)\right)+\underbrace{c}_{\text{Euler's\,\,constant}}
\end{align*}
}

\end{frame}





\begin{frame}
Solution method (for any given set of parameter values $\alpha$): 
\bigskip{}

\begin{itemize}
\itemsep1.5em
    \item<2-> Start at period $T$: $v_{ijT}(X_{iT}) = u_{ijT}(X_{iT})$
    \item<3-> Work backwards to period $t$
    \item<4-> Compute value functions at \textcolor{navy}{all possible states} $\{X_{i\tau}\}_{\tau=t}^{T}$
    \begin{itemize}
        \vspace{1.25em}
        \item<5-> Period $T$ typically has largest state space (even with Markov assumption)
    \end{itemize}
\end{itemize}

\bigskip{}

\onslide<6->{
\textcolor{navy}{What we solve for}: Policy functions---$\Pr(d_{it}=j)$ at all possible states
}

\onslide<7->{
\bigskip{}

\textcolor{navy}{Computational challenge}: State space grows with $T$
\bigskip{}

Various simplifying assumptions and workarounds can sidestep this challenge
}

\end{frame}


\end{document}