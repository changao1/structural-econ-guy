\documentclass[aspectratio=169]{beamer}

\usepackage{booktabs}
\usetheme{default}
\setbeamertemplate{navigation symbols}{}
\setbeamertemplate{itemize item}{\color{black}\textbullet}
\setbeamertemplate{itemize subitem}{\color{black}\textbullet}
\usepackage{xcolor}
\usepackage{tikz}
\usetikzlibrary{shapes,positioning,arrows}
\definecolor{navy}{RGB}{0, 0, 128}
\definecolor{lightblue}{RGB}{173, 216, 230}
\definecolor{lightgreen}{RGB}{144, 238, 144}




\begin{document}

\begin{frame}

To start with, let's discuss the theory of \textcolor{navy}{optimal stopping}

\bigskip{}

\onslide<2->{
Gives mathematical context for maximizing rewards or minimizing costs

\bigskip{}

Optimal stopping problems are by definition dynamic
}

\end{frame}

\begin{frame}

Many economic problems involve some sort of optimal stopping:

\begin{itemize}
\item[]
\item<2-> The Secretary Problem (when to hire from a sequence of job candidates)
\item[]
\item<3-> Search theory more generally (job search, spousal search, house search, ...)
\item[]
\item<4-> ``Buy/sell/hold'' problems (e.g. stock/options trading)
\item[]
\item<5-> Replacement problems (e.g. machines, infrastructure)
\end{itemize}

\end{frame}




\begin{frame}

Optimal stopping problems inherently have a tension between costs and benefits:

\begin{itemize}
\item[]
\item<2-> It is costly to interview job candidates
\item[]
\item<3-> But it is also costly to miss out on the best candidate
\end{itemize}

\end{frame}



\begin{frame}

In a discrete choice setting, \textcolor{navy}{dynamic programming} is the best solution method

\bigskip{}

\onslide<2->{
Within a discrete choice setting, time can be either continuous or discrete:
}

\begin{itemize}
\item[]
\item<3-> If continuous time: use Hamiltonians and Differential Equations
\item[]
\item<4-> If discrete time: use recursive methods
\end{itemize}
\end{frame}




\begin{frame}

Solution method also depends on the \textcolor{navy}{time horizon}:

\begin{itemize}
\item[]
\item<2-> If the time horizon is finite: then we can use dynamic programming
\item[]
\item<3-> If the time horizon is infinite: then need to (also) solve for a fixed point
\end{itemize}
\end{frame}


\begin{frame}

\onslide<1->{
All four combinations represent \textcolor{navy}{viable solution approaches} for discrete choice problems

\bigskip{}
}

\only<2>{
\begin{center}
\begin{tabular}{lcc}
\toprule
\textbf{} & \textbf{Finite Time Horizon} & \textbf{Infinite Time Horizon} \\
\midrule
\textbf{Continuous Time} & 
\begin{tabular}[c]{@{}c@{}}
Hamiltonians \& Diff. Eq., \\
Finite Differences
\end{tabular} & 
\begin{tabular}[c]{@{}c@{}}
Hamiltonians \& Diff. Eq., \\
Fin. Diff. \& Fixed Point
\end{tabular} \\
\hline
\textbf{Discrete Time} & 
\begin{tabular}[c]{@{}c@{}}
Dynamic Programming, \\
Backwards Recursion
\end{tabular} & 
\begin{tabular}[c]{@{}c@{}}
Dynamic Programming, \\
Bkw. Recursion \& Fixed Point
\end{tabular} \\
\bottomrule
\end{tabular}
\end{center}
}

\only<3>{
\begin{center}
\begin{tabular}{lcc}
\toprule
\textbf{} & \textbf{Finite Time Horizon} & \textbf{Infinite Time Horizon} \\
\midrule
\textbf{Continuous Time} & 
\phantom{\begin{tabular}[c]{@{}c@{}}
Hamiltonians \& Diff. Eq., \\
Finite Differences
\end{tabular}} & 
\phantom{\begin{tabular}[c]{@{}c@{}}
Hamiltonians \& Diff. Eq., \\
Fin. Diff. \& Fixed Point
\end{tabular}} \\
\hline
\textbf{Discrete Time} & 
\begin{tabular}[c]{@{}c@{}}
Dynamic Programming, \\
Backwards Recursion
\end{tabular} & 
\phantom{\begin{tabular}[c]{@{}c@{}}
Dynamic Programming, \\
Bkw. Recursion \& Fixed Point
\end{tabular}} \\
\bottomrule
\end{tabular}
\end{center}
}

    
\end{frame}


\end{document}