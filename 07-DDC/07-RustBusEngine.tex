\documentclass[aspectratio=169]{beamer}

\usetheme{default}
\setbeamertemplate{navigation symbols}{}
\setbeamertemplate{itemize item}{\color{black}\textbullet}
\setbeamertemplate{itemize subitem}{\color{black}\textbullet}
\usepackage{xcolor}
\definecolor{navy}{RGB}{0, 0, 128}

\begin{document}

\begin{frame}

Rust (1987) analyzes the decision to replace a bus engine $(d=1)$ or not $(d=0)$

\bigskip{}

\onslide<2->{
\textcolor{navy}{Why is this an optimal stopping problem?}
\bigskip{}

}

\begin{itemize}
\itemsep1.5em
\item<3-> Harold Zurcher wants to minimize operating costs
\item<4-> But unexpected bus breakdowns cause ``customer goodwill costs''
\item<5-> Premature replacement is costly, but so is breakdown
\item<6-> Goal: figure out when to optimally replace engines
\item<7-> Some buses get driven more than others
\end{itemize}

\end{frame}

\begin{frame}
Replacement decision depends on:
\bigskip{}

\begin{itemize}
\itemsep1.5em
\item Mileage on engine: $x_t$
\item Cost of replacing: $\overline{P}$
\item Scrap value of current engine: $\underline{P}$
\end{itemize}

\bigskip{}

\onslide<2->{
Payoffs net of error term:
\begin{align*}
u_0(x_{it},\theta) &= -c(x_{it},\theta)\\
u_1(x_{it},\theta) &= -[\overline{P}-\underline{P}+c(0,\theta)]
\end{align*}
}

\onslide<3->{
Mileage is discrete and transitions according to $f(x_{t+1}|x_t)$
}

\end{frame}



\begin{frame}

\textcolor{navy}{Why This Matters}

\bigskip{}

\begin{itemize}
\itemsep1.5em
\item<1-> Individual replacement decisions $\rightarrow$ aggregate investment volatility
\item<2-> Firms replacing equipment creates economy-wide boom/bust cycles
\item<3-> Wrong timing wastes billions in premature/delayed replacement
\item<4-> Transportation infrastructure especially critical for service quality
\end{itemize}

\end{frame}



\begin{frame}

\textcolor{navy}{Estimation Procedure}

\bigskip{}

\onslide<1->{
\textcolor{navy}{Step 1:} Calculate mileage transitions $f(x_{t+1}|x_t)$
}

\bigskip{}

\onslide<2->{
\textcolor{navy}{Step 2:} Maximize log likelihood of choices:
\begin{align*}
\ell(\theta;\mathbf{d},\mathbf{x}) = \sum_i\sum_t \sum_j (d_{it}=j)\log(p_{jt}(x_{it},\theta))
\end{align*}
}

\onslide<3->{
\textcolor{navy}{Step 3:} Within maximization, solve fixed point problem in the $v$'s each time log likelihood is evaluated
}

\end{frame}

\begin{frame}

\textcolor{navy}{Nested Fixed Point (NFXP) Algorithm}

\bigskip{}

\begin{itemize}
\itemsep1.5em
\item Outer loop: maximize likelihood over parameters $\theta$
\item Inner loop: solve for value functions $v$ given $\theta$
\item Repeat until convergence
\end{itemize}

\bigskip{}

\onslide<2->{
Each likelihood evaluation requires solving the dynamic programming problem
\bigskip{}

}

\onslide<3->{
Each outer loop iteration took 4 minutes ($N = 104$ buses, $T = 120$ months)
}

\end{frame}


\begin{frame}
What does Rust do with his results?
\bigskip{}

\begin{itemize}
\itemsep1.5em
    \item<1-> Examines different functional forms for $c(\cdot)$, including with unobs. heterogeneity
    \item<2-> Tests whether Zurcher behaves myopically (finds he doesn't)
    \item<3-> Shows that dynamic considerations result in less replacement at high mileage
    \item<4-> Infers implied demand for replacement as function of replacement cost
    \item<5-> Myopic demand curve is much more sensitive to replacement cost 
\end{itemize}
\end{frame}

\begin{frame}

\textcolor{navy}{Value of a structural model}
\bigskip{}

\begin{quote}
Since engine replacement costs have not varied much in the past, estimating replacement demand by a ``reduced-form'' approach which, for example, regresses engine replacements on replacement costs, is incapable of producing reliable estimates of the replacement demand function.

\bigskip{}

In terms of Figure 7, all the data would be clustered in a small ball about the intersection of the two demand curves: obviously many different demand functions would appear to fit the data equally well. 

\bigskip{}

The structural approach, on the other hand, efficiently concentrates additional information contained in the sequences $\{d_t, x_t\}$ into estimates of a small number of primitive parameters. Despite the relatively small number of such parameters, we obtain a rich behavioral model that can be used to answer a wide range of ``what if?'' policy questions.
\end{quote}    
\end{frame}


\end{document}