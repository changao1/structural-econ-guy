\documentclass[aspectratio=169]{beamer}

\usetheme{default}
\setbeamertemplate{navigation symbols}{}
\setbeamertemplate{itemize item}{\color{black}\textbullet}
\setbeamertemplate{itemize subitem}{\color{black}\textbullet}
\usepackage{xcolor}
\definecolor{navy}{RGB}{0, 0, 128}

\begin{document}

\begin{frame}

Rust (1987, footnote 12, original emphasis):

\bigskip{}

``The identification of $\beta$ depends on a priori specification of the utility function $u$. Actually $\beta$ is \textit{nonparametrically unidentified}: in the absence of a priori knowledge of the form of $u$ it is impossible to infer $\beta$.''

\bigskip{}

\onslide<2->{
``While this theoretical result might appear disturbing at first, on reflection it is clear we often do have substantial a priori information about $\beta$ itself. In the case of Zurcher, we know that $\beta$ must be `large' because $\beta=0$ implies an implausibly large rate of increase in monthly operating costs''
}

\end{frame}

\begin{frame}

The discount factor $\beta$ and utility parameters $\theta$ are confounded

\bigskip{}

\onslide<2->{
Two observationally equivalent explanations for behavior:
\bigskip{}

}

\begin{itemize}
\itemsep1.5em
    \item<3-> Low $\beta$ (impatient): ``I don't care about the future''
    \item<4-> High $\beta$ but low future utility: ``I care about the future, but it's not valuable to me''
\end{itemize}


\bigskip{}

\onslide<5->{
Without additional restrictions, these are indistinguishable in the data
}

\end{frame}

\begin{frame}

Rust's suggestion: impose strong a priori assumptions about $u(\cdot)$

\bigskip{}

\onslide<2->{
Example: Linear utility in maintenance costs
\begin{align*}
u(x_t, d_t) = -\theta c(x_t, d_t)
\end{align*}
}

\onslide<3->{
With known functional form, $\beta$ identified from intertemporal tradeoffs
}

\bigskip{}

\onslide<4->{
Problem: Strong assumptions may be incorrect
}

\bigskip{}

\onslide<5->{
``the difference in the log-likelihoods for $\beta=0$ vs. $\beta = .9999$ disappears as I generalize the specification of the cost function, c.''
}

\end{frame}

\begin{frame}
Alternative approach: find variables that affect transitions but not flow utility

\bigskip{}

\onslide<2->{
Need $Z_t$ such that:
\bigskip{}

\begin{itemize}
\itemsep1.5em
\item $Z_t$ affects $X_{t+1}$ given choice $d_t$  
\item $Z_t$ does not affect $u(X_t, d_t)$
\end{itemize}
}

\bigskip{}

\onslide<3->{
Intuition: $Z_t$ creates variation in future consequences without changing current payoffs
}

\bigskip{}

\onslide<4->{
Forward-looking agents respond to $Z_t$; myopic agents do not
}

\end{frame}

\begin{frame}

Empirical example: Arcidiacono, Sieg and Sloan (2007, IER)
\bigskip{}

\onslide<2->{
Studies smoking and drinking decisions of elderly
}

\bigskip{}

\onslide<3->{
Key idea: Age affects health transitions but not utility of consuming alcohol/tobacco
}

\bigskip{}

\onslide<4->{
Age excluded from utility function:
\begin{align*}
u(d_t, X_t) \neq f(\text{age}_t)
\end{align*}
}

\onslide<4->{
But age affects health transition probabilities:
\begin{align*}
P(X_{t+1}|\text{age}_t, d_t, X_t) \neq P(X_{t+1}| d_t, X_t)
\end{align*}
}

\end{frame}

\begin{frame}
How does this identify $\beta$?
\bigskip{}


\onslide<2->{
Age-consumption profiles reveal degree of forward-looking behavior
}

\bigskip{}

\onslide<3->{
High $\beta$ (forward-looking):
\bigskip{}

\begin{itemize}
\item Reduce harmful consumption with age due to shorter health horizon
\end{itemize}
}

\bigskip{}

\onslide<4->{
Low $\beta$ (myopic):
\bigskip{}

}

\begin{itemize}  
\itemsep1em
\item<5-> Consumption patterns driven by instantaneous utility only
\item<6-> Less responsive to age-related health risk changes
\end{itemize}

\bigskip{}

\onslide<7->{
Data determines which pattern fits better
}

\end{frame}

\begin{frame}

The main idea: leverage variables that create \textcolor{navy}{intertemporal wedges}

\bigskip{}

\onslide<2->{
Variables that change future consequences of current actions but not current payoffs
}

\bigskip{}

\onslide<3->{
Other examples:
\bigskip{}

\begin{itemize}
\itemsep1.5em
\item Policy changes affecting future (but not current) environment
\item Variation in information about future states
\item Individual differences in transition probabilities
\end{itemize}
}

\bigskip{}

\onslide<4->{
Forward-looking behavior reveals itself through responses to intertemporal wedges
}

\end{frame}

\end{document}